\begin{itemize}
\item INTRODUCTION: The Setting - bird eye's view - the challenge to be tackled / thing to be be improved in general
\item INTRODUCTION: Past research done
\item INTRODUCTION: Gap in knowledge/problem not yet solved
\item INTRODUCTION: Purpose and method of this work
\item INTRODUCTION: More detailed description what was done
\item INTRODUCTION: Results acquired
\item INTRODUCTION: Analysis and limitations of the results (Mostly relocate to Conclusions)
\item INTRODUCTION: Value (Mostly relocate to Conclusions)

\end{itemize}

The purpose of this thesis is to investigate the state of the art of signatureless detection methods of malicious network traffic. This type of traffic is typically generated by a piece of malware installed on a compromised machine. The prime example of malicious traffic we wish to detect is the so called beaconing traffic where the malware periodically contacts a command \& control server to check in and request tasks to execute. In this work we assume the point of view of a security analyst responsible for defending a network belonging to a reasonably sized company.

The best understood and currently most widely deployed approach to detect malware communications is using an intrusion detection system that leverages previously crafted signatures. This works quite well when facing known threats and information about them is being shared effectively between various defenders.

A more problematic scenario arises when tasked to detect a lesser known or perhaps even uniquely tailored piece of malware for which no pre-existing signatures are possible. In this work we will explore various proposed methods from previous research to categorise them and give an estimate on their feasibility and the repeatability of the claims they make. In order to facilitate this, a semi-automated testing system is set up, emulating a continuous integration work flow. This allows other contributors to easily add more detection methods and thus expand the breadth of the survey.

\iffalse

\section{Sample section with a table reference}

Ut hendrerit volutpat felis vitae aliquam. Duis quis augue urna. In sollicitudin lacinia elit, 
non ultrices dui tristique eu. In hac habitasse platea dictumst. Nullam mi sapien, sagittis non 
mi in, gravida lobortis ante. A sample latex table can be seen in Table~\ref{tab:sample_table}.


\begin{table}[!ht]
% Add some padding to the table cells:
\def\arraystretch{1.1}%
\begin{center}
  \caption{Sample table}
  \label{tab:sample_table}
  \begin{tabular}{| l | c | }
    \hline
    Sample & table \\
    \hline
    Sample & table \\
    Sample & table \\
    \hline
  \end{tabular}

  \end{center}
\end{table}

\section{Changelog}

\begin{itemize}
\item Changed \textit{\textbackslash{chapter's}} \textbackslash{newpage} to 
\textbackslash{clearpage} to prevent floats from wandering to the beginning of the next chapter

\item Added \textbf{[hyphens]} to the url package to prevent margin overflow with 
long urls

\item Added \textbf{multirow} package to make multirow and multicolumns possible

\item Added some helpful source code comments

\item Makefile for pdflatex and bibtex to automate pdf compilation

\item Abbreviations are autosorted by the Makefile

\item Added a bit of extra padding to the sample table
\end{itemize}

\subsection{Sample subsection with a figure reference}

Sed erat neque, cursus ac feugiat ac, sollicitudin
ut odio. Maecenas vel turpis rhoncus, euismod nisl ac, tincidunt ipsum. Curabitur fermentum vel
turpis ac lobortis. Cras a justo vitae diam volutpat blandit. Maecenas faucibus nibh a neque 
semper ullamcorper. Suspendisse in est vulputate, fermentum odio nec, pharetra augue. Fusce at
consequat arcu, sed hendrerit enim. Pellentesque id suscipit nibh, id pretium erat. 

Nam eget libero neque. Nullam commodo cursus turpis mollis cursus. Curabitur est tellus,
pellentesque eu velit sed, ullamcorper gravida felis. Proin vel cursus risus, at scelerisque 
justo. Quisque rutrum justo at ultricies auctor. A sample latex figure can be seen in
Figure~\ref{fig:oylogoe}. If your pictures appear grainy, you probably have too low dots
per inch (DPI) value.

\footnotetext[1]{Sample footnote}

%Pictures in .eps if you use latex, .pdf or .png if you use pdflatex. Don't specify the extension so you can use both!
\begin{figure}[ht]
  \begin{center}
    \includegraphics*[width=0.3\textwidth]{oylogoe}
  \end{center}
  \caption{A picture}
  \label{fig:oylogoe}
\end{figure}

\fi
